\documentclass[12pt]{article}
\usepackage{graphicx}

\title{Permutation Test for Autocorrelation in Annual Mean Temperature}
\begin{document}

\maketitle

\section*{Introduction}

Autocorrelation in climate time series is often present because measurements from
consecutive years are not independent. As a result, the usual p-value associated
with a Pearson correlation coefficient is not appropriate. To evaluate whether
annual mean temperatures in Key West exhibit significant correlation between
successive years, a permutation-based test was applied. This approach constructs
an empirical null distribution that does not rely on the assumption of temporal
independence.

\section*{Methods}

The correlation between temperatures in year $t$ and year $t+1$ was first
calculated using the original ordering of the annual mean temperature series.
This statistic served as the observed value.

To generate the null distribution, the temperature series was randomly permuted
10,000 times. For each permuted sequence, the correlation between adjacent
positions was recalculated. Because the order of years is destroyed in each
permutation, these values approximate the correlations expected when no temporal
dependence is present.

A one-sided permutation p-value was obtained as the fraction of permuted
correlation coefficients that were greater than or equal to the observed value.

\section*{Results}

Figure~\ref{fig:perm_hist} shows the null distribution of correlation coefficients obtained from
10,000 random permutations of the annual temperature series. The permutation process removes the temporal ordering,
so these coefficients reflect the range of statistical variation expected in the absence of temporal dependence.
The distribution is centred near zero, whereas the observed correlation from the original data (indicated by
the black vertical line) lies markedly to its right. This disparity suggests that such a value is highly unlikely
to arise under random permutations, thereby supporting the inference of a significant positive dependence between successive years.

\begin{figure}[h!]
\centering
\includegraphics[width=0.7\textwidth]{../results/TAutoCorr_null_hist.png}
\caption{Permutation distribution of adjacent-year correlation coefficients.
The vertical line marks the observed correlation calculated from the original
temperature series.}
\label{fig:perm_hist}
\end{figure}


\end{document}